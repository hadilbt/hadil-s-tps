\documentclass[12pt]{article}
\usepackage[utf8]{inputenc}
\usepackage[T1]{fontenc}
\usepackage[french]{babel}
\usepackage{geometry}
\geometry{a4paper, margin=2.5cm}
\usepackage{array}
\usepackage{longtable}

\title{Rapport de Planification des Adresses IP \\ 10/14/2025}
\author{}
\date{}

\begin{document}

\maketitle

\section*{Méthode de calcul du CIDR}
La formule utilisée pour déterminer le CIDR est : 
\[
\text{CIDR} = 32 - \log_2(\text{nombre d'hotes} + 2)
\]

\section*{Plan d'adressage IP par établissement}

\begin{longtable}{|>{\raggedright}p{3cm}|>{\raggedright}p{3cm}|>{\raggedright}p{3cm}|>{\raggedright}p{4cm}|>{\raggedright}p{3cm}|}
\hline
\textbf{Établissement} & \textbf{Nombre d'hôtes} & \textbf{CIDR / Masque} & \textbf{Plage d'adresses} & \textbf{Gateway / Broadcast} \\
\hline
\endfirsthead
\hline
\textbf{Établissement} & \textbf{Nombre d'hôtes} & \textbf{CIDR / Masque} & \textbf{Plage d'adresses} & \textbf{Gateway / Broadcast} \\
\hline
\endhead
\hline
\endfoot
\hline
\endlastfoot
EIDIA & 300 & /23, 255.255.254.0 & 10.15.0.1 $\rightarrow$ 10.15.1.254 & Gateway: 10.15.0.1, Broadcast: 10.15.1.255 \\
\hline
AD & 100 & /25, 255.255.255.128 & 10.15.2.1 $\rightarrow$ 10.15.2.126 & Gateway: 10.15.2.1, Broadcast: 10.15.2.127 \\
\hline
MED & 20 & /27, 255.255.255.224 & 10.15.2.129 $\rightarrow$ 10.15.2.158 & Gateway: 10.15.2.129, Broadcast: 10.15.2.158 \\
\hline
\end{longtable}

\end{document}
