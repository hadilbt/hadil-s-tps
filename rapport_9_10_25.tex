\documentclass[12pt]{article}

% --- Packages de base ---
\usepackage[utf8]{inputenc}
\usepackage[T1]{fontenc}
\usepackage[french]{babel}
\usepackage{geometry}
\usepackage{graphicx}
\usepackage{xcolor}
\usepackage{tcolorbox}
\usepackage{titlesec}
\usepackage{fancyhdr}
\usepackage{lmodern}
\usepackage{hyperref}

% --- Mise en page ---
\geometry{a4paper, margin=2.5cm}

% --- Couleurs modernes et douces ---
\definecolor{maincolor}{HTML}{1E88E5} % bleu clair
\definecolor{lightblue}{HTML}{E3F2FD}
\definecolor{textgray}{HTML}{333333}

% --- Police et couleur du texte ---
\renewcommand{\familydefault}{\sfdefault}
\color{textgray}

% --- Titres stylés ---
\titleformat{\section}
{\Large\bfseries\color{maincolor}}
{\thesection.}{1em}{}

% --- En-têtes et pieds de page stylés ---
\pagestyle{fancy}
\fancyhf{}
\fancyhead[L]{\color{maincolor}\textbf{Les commandes Git}}
\fancyhead[R]{\color{maincolor}Hadil Boutaouar}
\fancyfoot[C]{\color{maincolor}\thepage}
\renewcommand{\headrulewidth}{0.4pt}
\renewcommand{\footrulewidth}{0pt}

% --- Lien coloré ---
\hypersetup{
	colorlinks=true,
	linkcolor=maincolor,
	urlcolor=maincolor
}

% --- Style d'encadré pour les commandes ---
\tcbset{
	colback=lightblue,
	colframe=maincolor,
	boxrule=0.5pt,
	arc=3pt,
	left=6pt,
	right=6pt,
	top=4pt,
	bottom=4pt,
}

% --- Titre du document ---
\title{\textbf{Les commandes utilisées sur Git}}
\author{Hadil Boutaouar}
\date{\today}

\begin{document}
	
	% --- Page de titre simplifiée (sans image) ---
	\begin{titlepage}
		\centering
		\vspace*{3cm}
		{\Huge\color{maincolor}\textbf{Les commandes utilisées sur Git}}\\[1cm]
		{\Large Par : \textbf{Hadil Boutaouar}}\\[1cm]
		{\large \today}\\[3cm]
	
	\end{titlepage}
	
	\newpage
	
	\section{Introduction}
	Voici les 5 commandes principales utilisées par Hadil Boutaouar sur Git lors du TP de création de site web.
	
	\section{git init}
	La commande \texttt{git init} initialise un nouveau dépôt Git local.
	\begin{tcolorbox}
		git init
	\end{tcolorbox}
	\textbf{Utilité :} Elle permet de commencer à suivre les fichiers d’un projet avec Git.
	
	\section{git add}
	\texttt{git add} ajoute des fichiers à la zone de préparation avant un commit.
	\begin{tcolorbox}
		git add fichier.txt \\
		git add .
	\end{tcolorbox}
	\textbf{Utilité :} Sélectionner les fichiers à inclure dans le prochain commit.
	
	\section{git status}
	Affiche l’état actuel du dépôt Git (fichiers modifiés, ajoutés ou en attente de commit).
	\begin{tcolorbox}
		git status
	\end{tcolorbox}
	\textbf{Utilité :} Vérifie les changements avant de les enregistrer.
	
	\section{git commit}
	\texttt{git commit} enregistre les modifications dans l’historique du projet avec un message descriptif.
	\begin{tcolorbox}
		git commit -m "Ajout du fichier de configuration"
	\end{tcolorbox}
	\textbf{Utilité :} Crée une version stable et identifiable du projet.
	
	\section{git push}
	\texttt{git push} envoie les commits locaux vers un dépôt distant (comme GitHub).
	\begin{tcolorbox}
		git push origin main
	\end{tcolorbox}
	\textbf{Utilité :} Permet de partager les modifications avec d’autres collaborateurs.
	
\end{document}
